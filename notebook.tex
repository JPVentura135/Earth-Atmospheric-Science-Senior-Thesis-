
% Default to the notebook output style

    


% Inherit from the specified cell style.




    
\documentclass[11pt]{article}

    
    
    \usepackage[T1]{fontenc}
    % Nicer default font (+ math font) than Computer Modern for most use cases
    \usepackage{mathpazo}

    % Basic figure setup, for now with no caption control since it's done
    % automatically by Pandoc (which extracts ![](path) syntax from Markdown).
    \usepackage{graphicx}
    % We will generate all images so they have a width \maxwidth. This means
    % that they will get their normal width if they fit onto the page, but
    % are scaled down if they would overflow the margins.
    \makeatletter
    \def\maxwidth{\ifdim\Gin@nat@width>\linewidth\linewidth
    \else\Gin@nat@width\fi}
    \makeatother
    \let\Oldincludegraphics\includegraphics
    % Set max figure width to be 80% of text width, for now hardcoded.
    \renewcommand{\includegraphics}[1]{\Oldincludegraphics[width=.8\maxwidth]{#1}}
    % Ensure that by default, figures have no caption (until we provide a
    % proper Figure object with a Caption API and a way to capture that
    % in the conversion process - todo).
    \usepackage{caption}
    \DeclareCaptionLabelFormat{nolabel}{}
    \captionsetup{labelformat=nolabel}

    \usepackage{adjustbox} % Used to constrain images to a maximum size 
    \usepackage{xcolor} % Allow colors to be defined
    \usepackage{enumerate} % Needed for markdown enumerations to work
    \usepackage{geometry} % Used to adjust the document margins
    \usepackage{amsmath} % Equations
    \usepackage{amssymb} % Equations
    \usepackage{textcomp} % defines textquotesingle
    % Hack from http://tex.stackexchange.com/a/47451/13684:
    \AtBeginDocument{%
        \def\PYZsq{\textquotesingle}% Upright quotes in Pygmentized code
    }
    \usepackage{upquote} % Upright quotes for verbatim code
    \usepackage{eurosym} % defines \euro
    \usepackage[mathletters]{ucs} % Extended unicode (utf-8) support
    \usepackage[utf8x]{inputenc} % Allow utf-8 characters in the tex document
    \usepackage{fancyvrb} % verbatim replacement that allows latex
    \usepackage{grffile} % extends the file name processing of package graphics 
                         % to support a larger range 
    % The hyperref package gives us a pdf with properly built
    % internal navigation ('pdf bookmarks' for the table of contents,
    % internal cross-reference links, web links for URLs, etc.)
    \usepackage{hyperref}
    \usepackage{longtable} % longtable support required by pandoc >1.10
    \usepackage{booktabs}  % table support for pandoc > 1.12.2
    \usepackage[inline]{enumitem} % IRkernel/repr support (it uses the enumerate* environment)
    \usepackage[normalem]{ulem} % ulem is needed to support strikethroughs (\sout)
                                % normalem makes italics be italics, not underlines
    

    
    
    % Colors for the hyperref package
    \definecolor{urlcolor}{rgb}{0,.145,.698}
    \definecolor{linkcolor}{rgb}{.71,0.21,0.01}
    \definecolor{citecolor}{rgb}{.12,.54,.11}

    % ANSI colors
    \definecolor{ansi-black}{HTML}{3E424D}
    \definecolor{ansi-black-intense}{HTML}{282C36}
    \definecolor{ansi-red}{HTML}{E75C58}
    \definecolor{ansi-red-intense}{HTML}{B22B31}
    \definecolor{ansi-green}{HTML}{00A250}
    \definecolor{ansi-green-intense}{HTML}{007427}
    \definecolor{ansi-yellow}{HTML}{DDB62B}
    \definecolor{ansi-yellow-intense}{HTML}{B27D12}
    \definecolor{ansi-blue}{HTML}{208FFB}
    \definecolor{ansi-blue-intense}{HTML}{0065CA}
    \definecolor{ansi-magenta}{HTML}{D160C4}
    \definecolor{ansi-magenta-intense}{HTML}{A03196}
    \definecolor{ansi-cyan}{HTML}{60C6C8}
    \definecolor{ansi-cyan-intense}{HTML}{258F8F}
    \definecolor{ansi-white}{HTML}{C5C1B4}
    \definecolor{ansi-white-intense}{HTML}{A1A6B2}

    % commands and environments needed by pandoc snippets
    % extracted from the output of `pandoc -s`
    \providecommand{\tightlist}{%
      \setlength{\itemsep}{0pt}\setlength{\parskip}{0pt}}
    \DefineVerbatimEnvironment{Highlighting}{Verbatim}{commandchars=\\\{\}}
    % Add ',fontsize=\small' for more characters per line
    \newenvironment{Shaded}{}{}
    \newcommand{\KeywordTok}[1]{\textcolor[rgb]{0.00,0.44,0.13}{\textbf{{#1}}}}
    \newcommand{\DataTypeTok}[1]{\textcolor[rgb]{0.56,0.13,0.00}{{#1}}}
    \newcommand{\DecValTok}[1]{\textcolor[rgb]{0.25,0.63,0.44}{{#1}}}
    \newcommand{\BaseNTok}[1]{\textcolor[rgb]{0.25,0.63,0.44}{{#1}}}
    \newcommand{\FloatTok}[1]{\textcolor[rgb]{0.25,0.63,0.44}{{#1}}}
    \newcommand{\CharTok}[1]{\textcolor[rgb]{0.25,0.44,0.63}{{#1}}}
    \newcommand{\StringTok}[1]{\textcolor[rgb]{0.25,0.44,0.63}{{#1}}}
    \newcommand{\CommentTok}[1]{\textcolor[rgb]{0.38,0.63,0.69}{\textit{{#1}}}}
    \newcommand{\OtherTok}[1]{\textcolor[rgb]{0.00,0.44,0.13}{{#1}}}
    \newcommand{\AlertTok}[1]{\textcolor[rgb]{1.00,0.00,0.00}{\textbf{{#1}}}}
    \newcommand{\FunctionTok}[1]{\textcolor[rgb]{0.02,0.16,0.49}{{#1}}}
    \newcommand{\RegionMarkerTok}[1]{{#1}}
    \newcommand{\ErrorTok}[1]{\textcolor[rgb]{1.00,0.00,0.00}{\textbf{{#1}}}}
    \newcommand{\NormalTok}[1]{{#1}}
    
    % Additional commands for more recent versions of Pandoc
    \newcommand{\ConstantTok}[1]{\textcolor[rgb]{0.53,0.00,0.00}{{#1}}}
    \newcommand{\SpecialCharTok}[1]{\textcolor[rgb]{0.25,0.44,0.63}{{#1}}}
    \newcommand{\VerbatimStringTok}[1]{\textcolor[rgb]{0.25,0.44,0.63}{{#1}}}
    \newcommand{\SpecialStringTok}[1]{\textcolor[rgb]{0.73,0.40,0.53}{{#1}}}
    \newcommand{\ImportTok}[1]{{#1}}
    \newcommand{\DocumentationTok}[1]{\textcolor[rgb]{0.73,0.13,0.13}{\textit{{#1}}}}
    \newcommand{\AnnotationTok}[1]{\textcolor[rgb]{0.38,0.63,0.69}{\textbf{\textit{{#1}}}}}
    \newcommand{\CommentVarTok}[1]{\textcolor[rgb]{0.38,0.63,0.69}{\textbf{\textit{{#1}}}}}
    \newcommand{\VariableTok}[1]{\textcolor[rgb]{0.10,0.09,0.49}{{#1}}}
    \newcommand{\ControlFlowTok}[1]{\textcolor[rgb]{0.00,0.44,0.13}{\textbf{{#1}}}}
    \newcommand{\OperatorTok}[1]{\textcolor[rgb]{0.40,0.40,0.40}{{#1}}}
    \newcommand{\BuiltInTok}[1]{{#1}}
    \newcommand{\ExtensionTok}[1]{{#1}}
    \newcommand{\PreprocessorTok}[1]{\textcolor[rgb]{0.74,0.48,0.00}{{#1}}}
    \newcommand{\AttributeTok}[1]{\textcolor[rgb]{0.49,0.56,0.16}{{#1}}}
    \newcommand{\InformationTok}[1]{\textcolor[rgb]{0.38,0.63,0.69}{\textbf{\textit{{#1}}}}}
    \newcommand{\WarningTok}[1]{\textcolor[rgb]{0.38,0.63,0.69}{\textbf{\textit{{#1}}}}}
    
    
    % Define a nice break command that doesn't care if a line doesn't already
    % exist.
    \def\br{\hspace*{\fill} \\* }
    % Math Jax compatability definitions
    \def\gt{>}
    \def\lt{<}
    % Document parameters
    \title{ES\_Thesis\_work}
    
    
    

    % Pygments definitions
    
\makeatletter
\def\PY@reset{\let\PY@it=\relax \let\PY@bf=\relax%
    \let\PY@ul=\relax \let\PY@tc=\relax%
    \let\PY@bc=\relax \let\PY@ff=\relax}
\def\PY@tok#1{\csname PY@tok@#1\endcsname}
\def\PY@toks#1+{\ifx\relax#1\empty\else%
    \PY@tok{#1}\expandafter\PY@toks\fi}
\def\PY@do#1{\PY@bc{\PY@tc{\PY@ul{%
    \PY@it{\PY@bf{\PY@ff{#1}}}}}}}
\def\PY#1#2{\PY@reset\PY@toks#1+\relax+\PY@do{#2}}

\expandafter\def\csname PY@tok@w\endcsname{\def\PY@tc##1{\textcolor[rgb]{0.73,0.73,0.73}{##1}}}
\expandafter\def\csname PY@tok@c\endcsname{\let\PY@it=\textit\def\PY@tc##1{\textcolor[rgb]{0.25,0.50,0.50}{##1}}}
\expandafter\def\csname PY@tok@cp\endcsname{\def\PY@tc##1{\textcolor[rgb]{0.74,0.48,0.00}{##1}}}
\expandafter\def\csname PY@tok@k\endcsname{\let\PY@bf=\textbf\def\PY@tc##1{\textcolor[rgb]{0.00,0.50,0.00}{##1}}}
\expandafter\def\csname PY@tok@kp\endcsname{\def\PY@tc##1{\textcolor[rgb]{0.00,0.50,0.00}{##1}}}
\expandafter\def\csname PY@tok@kt\endcsname{\def\PY@tc##1{\textcolor[rgb]{0.69,0.00,0.25}{##1}}}
\expandafter\def\csname PY@tok@o\endcsname{\def\PY@tc##1{\textcolor[rgb]{0.40,0.40,0.40}{##1}}}
\expandafter\def\csname PY@tok@ow\endcsname{\let\PY@bf=\textbf\def\PY@tc##1{\textcolor[rgb]{0.67,0.13,1.00}{##1}}}
\expandafter\def\csname PY@tok@nb\endcsname{\def\PY@tc##1{\textcolor[rgb]{0.00,0.50,0.00}{##1}}}
\expandafter\def\csname PY@tok@nf\endcsname{\def\PY@tc##1{\textcolor[rgb]{0.00,0.00,1.00}{##1}}}
\expandafter\def\csname PY@tok@nc\endcsname{\let\PY@bf=\textbf\def\PY@tc##1{\textcolor[rgb]{0.00,0.00,1.00}{##1}}}
\expandafter\def\csname PY@tok@nn\endcsname{\let\PY@bf=\textbf\def\PY@tc##1{\textcolor[rgb]{0.00,0.00,1.00}{##1}}}
\expandafter\def\csname PY@tok@ne\endcsname{\let\PY@bf=\textbf\def\PY@tc##1{\textcolor[rgb]{0.82,0.25,0.23}{##1}}}
\expandafter\def\csname PY@tok@nv\endcsname{\def\PY@tc##1{\textcolor[rgb]{0.10,0.09,0.49}{##1}}}
\expandafter\def\csname PY@tok@no\endcsname{\def\PY@tc##1{\textcolor[rgb]{0.53,0.00,0.00}{##1}}}
\expandafter\def\csname PY@tok@nl\endcsname{\def\PY@tc##1{\textcolor[rgb]{0.63,0.63,0.00}{##1}}}
\expandafter\def\csname PY@tok@ni\endcsname{\let\PY@bf=\textbf\def\PY@tc##1{\textcolor[rgb]{0.60,0.60,0.60}{##1}}}
\expandafter\def\csname PY@tok@na\endcsname{\def\PY@tc##1{\textcolor[rgb]{0.49,0.56,0.16}{##1}}}
\expandafter\def\csname PY@tok@nt\endcsname{\let\PY@bf=\textbf\def\PY@tc##1{\textcolor[rgb]{0.00,0.50,0.00}{##1}}}
\expandafter\def\csname PY@tok@nd\endcsname{\def\PY@tc##1{\textcolor[rgb]{0.67,0.13,1.00}{##1}}}
\expandafter\def\csname PY@tok@s\endcsname{\def\PY@tc##1{\textcolor[rgb]{0.73,0.13,0.13}{##1}}}
\expandafter\def\csname PY@tok@sd\endcsname{\let\PY@it=\textit\def\PY@tc##1{\textcolor[rgb]{0.73,0.13,0.13}{##1}}}
\expandafter\def\csname PY@tok@si\endcsname{\let\PY@bf=\textbf\def\PY@tc##1{\textcolor[rgb]{0.73,0.40,0.53}{##1}}}
\expandafter\def\csname PY@tok@se\endcsname{\let\PY@bf=\textbf\def\PY@tc##1{\textcolor[rgb]{0.73,0.40,0.13}{##1}}}
\expandafter\def\csname PY@tok@sr\endcsname{\def\PY@tc##1{\textcolor[rgb]{0.73,0.40,0.53}{##1}}}
\expandafter\def\csname PY@tok@ss\endcsname{\def\PY@tc##1{\textcolor[rgb]{0.10,0.09,0.49}{##1}}}
\expandafter\def\csname PY@tok@sx\endcsname{\def\PY@tc##1{\textcolor[rgb]{0.00,0.50,0.00}{##1}}}
\expandafter\def\csname PY@tok@m\endcsname{\def\PY@tc##1{\textcolor[rgb]{0.40,0.40,0.40}{##1}}}
\expandafter\def\csname PY@tok@gh\endcsname{\let\PY@bf=\textbf\def\PY@tc##1{\textcolor[rgb]{0.00,0.00,0.50}{##1}}}
\expandafter\def\csname PY@tok@gu\endcsname{\let\PY@bf=\textbf\def\PY@tc##1{\textcolor[rgb]{0.50,0.00,0.50}{##1}}}
\expandafter\def\csname PY@tok@gd\endcsname{\def\PY@tc##1{\textcolor[rgb]{0.63,0.00,0.00}{##1}}}
\expandafter\def\csname PY@tok@gi\endcsname{\def\PY@tc##1{\textcolor[rgb]{0.00,0.63,0.00}{##1}}}
\expandafter\def\csname PY@tok@gr\endcsname{\def\PY@tc##1{\textcolor[rgb]{1.00,0.00,0.00}{##1}}}
\expandafter\def\csname PY@tok@ge\endcsname{\let\PY@it=\textit}
\expandafter\def\csname PY@tok@gs\endcsname{\let\PY@bf=\textbf}
\expandafter\def\csname PY@tok@gp\endcsname{\let\PY@bf=\textbf\def\PY@tc##1{\textcolor[rgb]{0.00,0.00,0.50}{##1}}}
\expandafter\def\csname PY@tok@go\endcsname{\def\PY@tc##1{\textcolor[rgb]{0.53,0.53,0.53}{##1}}}
\expandafter\def\csname PY@tok@gt\endcsname{\def\PY@tc##1{\textcolor[rgb]{0.00,0.27,0.87}{##1}}}
\expandafter\def\csname PY@tok@err\endcsname{\def\PY@bc##1{\setlength{\fboxsep}{0pt}\fcolorbox[rgb]{1.00,0.00,0.00}{1,1,1}{\strut ##1}}}
\expandafter\def\csname PY@tok@kc\endcsname{\let\PY@bf=\textbf\def\PY@tc##1{\textcolor[rgb]{0.00,0.50,0.00}{##1}}}
\expandafter\def\csname PY@tok@kd\endcsname{\let\PY@bf=\textbf\def\PY@tc##1{\textcolor[rgb]{0.00,0.50,0.00}{##1}}}
\expandafter\def\csname PY@tok@kn\endcsname{\let\PY@bf=\textbf\def\PY@tc##1{\textcolor[rgb]{0.00,0.50,0.00}{##1}}}
\expandafter\def\csname PY@tok@kr\endcsname{\let\PY@bf=\textbf\def\PY@tc##1{\textcolor[rgb]{0.00,0.50,0.00}{##1}}}
\expandafter\def\csname PY@tok@bp\endcsname{\def\PY@tc##1{\textcolor[rgb]{0.00,0.50,0.00}{##1}}}
\expandafter\def\csname PY@tok@fm\endcsname{\def\PY@tc##1{\textcolor[rgb]{0.00,0.00,1.00}{##1}}}
\expandafter\def\csname PY@tok@vc\endcsname{\def\PY@tc##1{\textcolor[rgb]{0.10,0.09,0.49}{##1}}}
\expandafter\def\csname PY@tok@vg\endcsname{\def\PY@tc##1{\textcolor[rgb]{0.10,0.09,0.49}{##1}}}
\expandafter\def\csname PY@tok@vi\endcsname{\def\PY@tc##1{\textcolor[rgb]{0.10,0.09,0.49}{##1}}}
\expandafter\def\csname PY@tok@vm\endcsname{\def\PY@tc##1{\textcolor[rgb]{0.10,0.09,0.49}{##1}}}
\expandafter\def\csname PY@tok@sa\endcsname{\def\PY@tc##1{\textcolor[rgb]{0.73,0.13,0.13}{##1}}}
\expandafter\def\csname PY@tok@sb\endcsname{\def\PY@tc##1{\textcolor[rgb]{0.73,0.13,0.13}{##1}}}
\expandafter\def\csname PY@tok@sc\endcsname{\def\PY@tc##1{\textcolor[rgb]{0.73,0.13,0.13}{##1}}}
\expandafter\def\csname PY@tok@dl\endcsname{\def\PY@tc##1{\textcolor[rgb]{0.73,0.13,0.13}{##1}}}
\expandafter\def\csname PY@tok@s2\endcsname{\def\PY@tc##1{\textcolor[rgb]{0.73,0.13,0.13}{##1}}}
\expandafter\def\csname PY@tok@sh\endcsname{\def\PY@tc##1{\textcolor[rgb]{0.73,0.13,0.13}{##1}}}
\expandafter\def\csname PY@tok@s1\endcsname{\def\PY@tc##1{\textcolor[rgb]{0.73,0.13,0.13}{##1}}}
\expandafter\def\csname PY@tok@mb\endcsname{\def\PY@tc##1{\textcolor[rgb]{0.40,0.40,0.40}{##1}}}
\expandafter\def\csname PY@tok@mf\endcsname{\def\PY@tc##1{\textcolor[rgb]{0.40,0.40,0.40}{##1}}}
\expandafter\def\csname PY@tok@mh\endcsname{\def\PY@tc##1{\textcolor[rgb]{0.40,0.40,0.40}{##1}}}
\expandafter\def\csname PY@tok@mi\endcsname{\def\PY@tc##1{\textcolor[rgb]{0.40,0.40,0.40}{##1}}}
\expandafter\def\csname PY@tok@il\endcsname{\def\PY@tc##1{\textcolor[rgb]{0.40,0.40,0.40}{##1}}}
\expandafter\def\csname PY@tok@mo\endcsname{\def\PY@tc##1{\textcolor[rgb]{0.40,0.40,0.40}{##1}}}
\expandafter\def\csname PY@tok@ch\endcsname{\let\PY@it=\textit\def\PY@tc##1{\textcolor[rgb]{0.25,0.50,0.50}{##1}}}
\expandafter\def\csname PY@tok@cm\endcsname{\let\PY@it=\textit\def\PY@tc##1{\textcolor[rgb]{0.25,0.50,0.50}{##1}}}
\expandafter\def\csname PY@tok@cpf\endcsname{\let\PY@it=\textit\def\PY@tc##1{\textcolor[rgb]{0.25,0.50,0.50}{##1}}}
\expandafter\def\csname PY@tok@c1\endcsname{\let\PY@it=\textit\def\PY@tc##1{\textcolor[rgb]{0.25,0.50,0.50}{##1}}}
\expandafter\def\csname PY@tok@cs\endcsname{\let\PY@it=\textit\def\PY@tc##1{\textcolor[rgb]{0.25,0.50,0.50}{##1}}}

\def\PYZbs{\char`\\}
\def\PYZus{\char`\_}
\def\PYZob{\char`\{}
\def\PYZcb{\char`\}}
\def\PYZca{\char`\^}
\def\PYZam{\char`\&}
\def\PYZlt{\char`\<}
\def\PYZgt{\char`\>}
\def\PYZsh{\char`\#}
\def\PYZpc{\char`\%}
\def\PYZdl{\char`\$}
\def\PYZhy{\char`\-}
\def\PYZsq{\char`\'}
\def\PYZdq{\char`\"}
\def\PYZti{\char`\~}
% for compatibility with earlier versions
\def\PYZat{@}
\def\PYZlb{[}
\def\PYZrb{]}
\makeatother


    % Exact colors from NB
    \definecolor{incolor}{rgb}{0.0, 0.0, 0.5}
    \definecolor{outcolor}{rgb}{0.545, 0.0, 0.0}



    
    % Prevent overflowing lines due to hard-to-break entities
    \sloppy 
    % Setup hyperref package
    \hypersetup{
      breaklinks=true,  % so long urls are correctly broken across lines
      colorlinks=true,
      urlcolor=urlcolor,
      linkcolor=linkcolor,
      citecolor=citecolor,
      }
    % Slightly bigger margins than the latex defaults
    
    \geometry{verbose,tmargin=1in,bmargin=1in,lmargin=1in,rmargin=1in}
    
    

    \begin{document}
    
    
    \maketitle
    
    

    
    Jean-paul Ventura\\
City University of New York, Hunter College.\\
Advisor: Randye L. Rutberg.

    \section{Earth Science Senior Thesis
Work:}\label{earth-science-senior-thesis-work}

\subsection{On the role of the of the carbon cycle and plate tectonics
in planetary habitability. Earth as an
analog.}\label{on-the-role-of-the-of-the-carbon-cycle-and-plate-tectonics-in-planetary-habitability.-earth-as-an-analog.}

    In the first order discussion about the habitability of an environment
or planet, temperature is the paramount quantity to investigate. This is
because, under the current description of planetary habitability, a
temperature regime which supports liquid water is understood to be a
necessary requirement for the potential for life. For a terrestrial
analog, temperature via stellar flux and surface water content will
determine the exposed surface land fraction, which influences planetary
surface weathering behavior. This is parameter is important because the
silicate-weathering feedback, which is governed by this land fraction,
determines the width of the habitability zone in both space and time.
This feedback mechanism serves as a control on atmospheric carbon over
geologic timescales (\textgreater{}Myr) and is characterized by the
dissolution of atmospheric carbon in precipitation and subsequent uptake
in exposed silicate rock via chemical reaction between carbonic acid in
the rainwater and minerals in the silicate. Furthermore, in addition to
stellar flux and distance from the host star, surface and geologic
effects on the planetary body are of crucial importance in determining
the suitability for life based on their effect on surface temperature.

    \subsubsection{Introduction}\label{introduction}

    The habitable zone (HZ) is traditionally defined as the region around a
star where liquid water can exist in stable phase at the surface of a
planet. (Kasting et al 1993). Since climate systems include both
positive and negative feedbacks which also define the planetary surface
temperature, a planets surface temperature is non-trivially related to
potential habitability. The inner orbital distances of the HZ are
defined as the "moist greenhouse" region, characterized by the planet
becoming hot enough (surface temperature \textasciitilde{}340K) that
large amounts of water can be lost by photolysis in the stratosphere and
subsequent hydrogen escape to space. The outer edge of the zone is set
where CO2 reaches a high enough pressure that it can no longer provide
warming, either because of increased Rayleigh scattering or because it
condenses at the surface, which then results in permanent global
glaciation. These limits, however, do not represent strict barriers to
all types of life. For example, on Earth, life survived glaciations that
may have been global, a period known as "snowball earth" ocurring
600-700 million years ago. (kirschvink 1992;Hoffman et al, 1998.) ... In
all, to begin to be able to interpret the sensitivities provided by
climate feedbacks, a model of steady-state planetary energy balance must
be employed.

    \begin{Verbatim}[commandchars=\\\{\}]
{\color{incolor}In [{\color{incolor}1}]:} \PY{k+kn}{import} \PY{n+nn}{matplotlib}\PY{n+nn}{.}\PY{n+nn}{pyplot} \PY{k}{as} \PY{n+nn}{plt}
        \PY{k+kn}{import} \PY{n+nn}{astropy}\PY{n+nn}{.}\PY{n+nn}{units} \PY{k}{as} \PY{n+nn}{u}
        \PY{k+kn}{import} \PY{n+nn}{numpy} \PY{k}{as} \PY{n+nn}{np}
\end{Verbatim}


    \begin{Verbatim}[commandchars=\\\{\}]
{\color{incolor}In [{\color{incolor}2}]:} \PY{n}{Albedo} \PY{o}{=} \PY{l+m+mf}{0.7} \PY{c+c1}{\PYZsh{} unitless quantity, surface reflectivity of a body expressed as a fraction of total stellar output.}
\end{Verbatim}


    \paragraph{A closer look at planetary energy
balance}\label{a-closer-look-at-planetary-energy-balance}

    To first order, Earths temperature depends on 3 factors:

1.) The stellar flux available at the distance of earths orbit.\\
2.) Earths reflectivity (AKA albedo {[}\%{]})\\
3.) The amount of warming provided by the atmosphere (Greenhouse
effects).

Stellar flux, S, is the amount of solar energy reaching the top of
Earths atmosphere. Not all of this energy is is absorbed, however.
\$\approx\$30\% is reflected, mostly by clouds. This reflectivity or
albedo, is usually represented as a fraction of the total incident
starlight that is reflected from the planet as a whole. To calculate
3.), we treat Earth as a blackbody (although this is of course an
approximation not considering that the atmosphere radiates better at
some wavelength than others because of the thermal and optical opacity
of gases like H20 and CO2). We start by defining \(T_{eff}\), the
effective radiating temperature of the planet or the energy that a true
blackbody would need to radiate the amount of energy that earth does.

We then use the Stefan-Boltzman equation, which describes the power
radiated from a body of mass based on its temperature, to calculate the
energy emitted by Earth and set it equal to the incident energy from the
sun.

    \paragraph{Modeling Planetary Energy
Balance}\label{modeling-planetary-energy-balance}

    In order to start modeling planetary energy balance, we start with a
simple equation regarding a steady state system:

\(E_{emitted}\) \(=\) \(E_{absorbed}\) \(\space\space\space\space\)
\((i)\)

Treating Earth as a blackbody with an effectve radiating temperature,
\(T_{eff}\), The Stefan-Boltzmann law tells us that for a blackbody, in
this case, the Earth:

\(E_{emitted}\) \(=\) \(4\pi R_{Earth}^2\) \(\times\) \(\sigma\)
\(T_{eff}^4\)

Now for the energy absorbed by Earth:

From the sun, Earth would look like a circle whose area is
\(\pi R_{Earth}^2\). Note that this is an area of Earth as a disk. We
make this approximation since not all rays from the sun strike the Earth
surface perpendicularly. Thus,

\(Energy \space absorbed \space by \space Earth\) \(=\)
\(Energy \space intercepted\) - \(Energy \space reflected\)

\(=\) \((\pi R_{Earth}^2 \times S)\) \(-\)
\((\pi R_{Earth}^2 \times S \times A)\)

\(=\) \(\pi R_{Earth}^2 \times S(1-A)\) \(\space\space\space\space\)
\((ii)\)

Where the stellar flux, \(S\), is given by:

\(S\) \(=\) \(S_{o}\) \((\frac{r_{o}}{r})^2\)

where \(r_{o}\) is the radius of

\begin{description}
\item[Revisiting equation \((i)\) and equating with \((ii)\), we have]
\end{description}

\(4\pi R_{Earth}^2\) \(\times\) \(\sigma\) \(T_{eff}^4\) \(=\)
\(\pi R_{Earth}^2 \times S(1-A)\)

\(\therefore\) \(\space\space\) \(\sigma\) \(T_{eff}^4\) \(=\)
\(\frac{S}{4}(1-A)\)

    \paragraph{Computing the circumstellar habitability zone around a main
sequence star. (dM = 3.0
)}\label{computing-the-circumstellar-habitability-zone-around-a-main-sequence-star.-dm-3.0}

    To compute the orbital distances in which water on the surface of a
terrestrial planet can exist in stable phase we start with estimating
the stars absolute visual magnitude, \(M_{v}\), based on apparent visual
magnitude, \(m_{v}\).

    \(M_{v}\) \(=\) \(m_{v}\) \(-\) \(5\times log(\frac{d}{10})\)

    Where:

\(M_{v}\) \(=\) absolute visual magnitude of the star

\(m_{v}\) \(=\) apparent visual magnitude of the star

\(d\) \(\space\) \(=\) distance from Earth to the star in parsecs

    Using values for the M-dwarf Gliese 581:

    \begin{Verbatim}[commandchars=\\\{\}]
{\color{incolor}In [{\color{incolor}3}]:} \PY{n}{m\PYZus{}v} \PY{o}{=} \PY{l+m+mf}{10.55}
        \PY{n}{d}   \PY{o}{=} \PY{l+m+mf}{6.26} \PY{c+c1}{\PYZsh{} parsecs}
        
        \PY{n}{M\PYZus{}v} \PY{o}{=} \PY{n}{m\PYZus{}v} \PY{o}{\PYZhy{}} \PY{p}{(}\PY{l+m+mf}{5.0} \PY{o}{*} \PY{n}{np}\PY{o}{.}\PY{n}{log10}\PY{p}{(}\PY{n}{d}\PY{o}{/}\PY{l+m+mi}{10}\PY{p}{)}\PY{p}{)} 
        
        \PY{n+nb}{print}\PY{p}{(}\PY{l+s+s1}{\PYZsq{}}\PY{l+s+s1}{Absolute visual magnitude of Gliese 681: }\PY{l+s+s1}{\PYZsq{}}\PY{p}{,} \PY{n}{M\PYZus{}v}\PY{p}{)}
\end{Verbatim}


    \begin{Verbatim}[commandchars=\\\{\}]
Absolute visual magnitude of Gliese 681:  11.567128333947853

    \end{Verbatim}

    We then compute the bolometric magnitude, \(M_{bol}\), of the host star:

    \(M_{bol}\) \(=\) \(M_v + BC\)

    Where:

    \(BC\) \(=\) \(Bolometric\) \(correction\) \(constant\)

    \begin{Verbatim}[commandchars=\\\{\}]
{\color{incolor}In [{\color{incolor}4}]:} \PY{n}{BC} \PY{o}{=} \PY{o}{\PYZhy{}}\PY{l+m+mf}{2.0} \PY{c+c1}{\PYZsh{} dimensionless quantity for the M spectral class }
        
        \PY{n}{M\PYZus{}bol} \PY{o}{=} \PY{n}{M\PYZus{}v} \PY{o}{+} \PY{n}{BC}
        
        \PY{n+nb}{print}\PY{p}{(}\PY{l+s+s1}{\PYZsq{}}\PY{l+s+s1}{The bolometric magnitude of Gliese 681 is: }\PY{l+s+s1}{\PYZsq{}}\PY{p}{,}\PY{n}{M\PYZus{}bol}\PY{p}{)}
\end{Verbatim}


    \begin{Verbatim}[commandchars=\\\{\}]
The bolometric magnitude of Gliese 681 is:  9.567128333947853

    \end{Verbatim}

    Now using the bolometric magnitude, \(M_{bol}\), we can calculate an
absolute luminosity, \(L_{sol}\), and then define the boundaries of the
habitable zone.

    \(\Large L_{abs}\) \(=\) \(\Large\frac{L_{Gl 581}}{L_{sol}}\) \(=\)
\(\Large10^\frac{M_{bol}-M_{sol}}{P.R.}\)

    where:

    \(M_{sol}\) \(=\) the absolute visual magnitude of the sun.

    \(P.R\) \(=\) \(The\) \(Pogson\) \(Ratio\), \(a\) \(factor\) \(which\)
\(defines\) \(the\) \(difference\) \(in\) \(brightness\) \(between\)
\(two\) \(stars\) \(of\) \(differing\) \(magnitude\)

    \begin{Verbatim}[commandchars=\\\{\}]
{\color{incolor}In [{\color{incolor}7}]:} \PY{n}{M\PYZus{}sol} \PY{o}{=} \PY{l+m+mf}{4.72}
        
        \PY{n}{abs\PYZus{}luminosity} \PY{o}{=} \PY{l+m+mi}{10}\PY{o}{*}\PY{o}{*}\PY{p}{(}\PY{p}{(}\PY{n}{M\PYZus{}bol}\PY{o}{\PYZhy{}}\PY{n}{M\PYZus{}sol}\PY{p}{)}\PY{o}{/}\PY{p}{(}\PY{o}{\PYZhy{}}\PY{l+m+mf}{2.5}\PY{p}{)}\PY{p}{)}
        
        \PY{n+nb}{print}\PY{p}{(}\PY{l+s+s1}{\PYZsq{}}\PY{l+s+s1}{Absolute luminosity of Gliese 681: }\PY{l+s+s1}{\PYZsq{}}\PY{p}{,} \PY{n}{abs\PYZus{}luminosity}\PY{p}{)}
\end{Verbatim}


    \begin{Verbatim}[commandchars=\\\{\}]
Absolute luminosity of Gliese 681:  0.0115119439501

    \end{Verbatim}

    \paragraph{Approximating the boundaries of the circumstellar
habitability
zone.}\label{approximating-the-boundaries-of-the-circumstellar-habitability-zone.}

    \(\large r_{inner}\) \(=\) \(\large \sqrt{\frac{L_{abs}}{-1.1}}\)

    \(\large r_{outer}\) \(=\) \(\large \sqrt{\frac{L_{abs}}{0.53}}\)

    \begin{Verbatim}[commandchars=\\\{\}]
{\color{incolor}In [{\color{incolor}25}]:} \PY{n}{r\PYZus{}inner} \PY{o}{=} \PY{n}{np}\PY{o}{.}\PY{n}{sqrt}\PY{p}{(}\PY{p}{(}\PY{n}{abs\PYZus{}luminosity}\PY{o}{/}\PY{l+m+mf}{1.1}\PY{p}{)}\PY{p}{)} \PY{o}{*} \PY{n}{u}\PY{o}{.}\PY{n}{AU}
         \PY{n}{r\PYZus{}inner} 
\end{Verbatim}

\texttt{\color{outcolor}Out[{\color{outcolor}25}]:}
    
    $0.10230056 \; \mathrm{AU}$

    

    \begin{Verbatim}[commandchars=\\\{\}]
{\color{incolor}In [{\color{incolor}28}]:} \PY{n}{r\PYZus{}outer} \PY{o}{=} \PY{n}{np}\PY{o}{.}\PY{n}{sqrt}\PY{p}{(}\PY{p}{(}\PY{n}{abs\PYZus{}luminosity}\PY{o}{/}\PY{l+m+mf}{0.53}\PY{p}{)}\PY{p}{)} \PY{o}{*} \PY{n}{u}\PY{o}{.}\PY{n}{AU}
         \PY{n}{r\PYZus{}outer} 
\end{Verbatim}

\texttt{\color{outcolor}Out[{\color{outcolor}28}]:}
    
    $0.14737927 \; \mathrm{AU}$

    

    Having aquired the orbital range for our habitability zone, lets return
to calculating and interpreting planetary effective temperature.

    \subsubsection{Computing the effective radiating temperature of a
terrestrial analog around an Early M
dwarf.}\label{computing-the-effective-radiating-temperature-of-a-terrestrial-analog-around-an-early-m-dwarf.}

    \(S\) \(=\) \(S_{o}\) \((\frac{r_{o}}{r})^2\)

    \begin{Verbatim}[commandchars=\\\{\}]
{\color{incolor}In [{\color{incolor} }]:} \PY{n}{S} \PY{o}{=} \PY{l+m+mi}{1366}
\end{Verbatim}


    \(T_{eff}\) \(=\) \(\sqrt[4]{\frac{S}{4\sigma}(1-A)}\)

    \section{\texorpdfstring{\textsuperscript{\^{}} Develop explanation for
this section
\textsuperscript{\^{}}}{\^{} Develop explanation for this section \^{}}}\label{develop-explanation-for-this-section}

    \begin{Verbatim}[commandchars=\\\{\}]
{\color{incolor}In [{\color{incolor} }]:} \PY{n}{t\PYZus{}eff} \PY{o}{=} \PY{p}{[}\PY{p}{]}
        \PY{n}{orb\PYZus{}radii} \PY{o}{=} \PY{p}{[}\PY{l+m+mf}{0.05}\PY{p}{,} \PY{l+m+mf}{0.10}\PY{p}{,} \PY{l+m+mf}{0.15}\PY{p}{,} \PY{l+m+mf}{0.20}\PY{p}{,} \PY{l+m+mf}{0.25}\PY{p}{,} \PY{l+m+mf}{0.30}\PY{p}{,} \PY{l+m+mf}{0.35}\PY{p}{,} \PY{l+m+mf}{0.40}\PY{p}{,} \PY{l+m+mf}{0.45}\PY{p}{,} \PY{l+m+mf}{0.50}\PY{p}{,} \PY{l+m+mf}{0.55}\PY{p}{,} \PY{l+m+mf}{0.60}\PY{p}{,} \PY{l+m+mf}{0.65}\PY{p}{]}
        
        \PY{k}{for} \PY{n}{radius} \PY{o+ow}{in} \PY{n}{orb\PYZus{}radii}\PY{p}{:}
            \PY{n}{teff} \PY{o}{=} 
            
            
\end{Verbatim}



    % Add a bibliography block to the postdoc
    
    
    
    \end{document}
